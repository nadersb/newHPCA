%%%%%%%%%%%%%%%%%%%%%%%%%%%%%%%%%%%%
% This is the template for submission to HPCA 2019
% The cls file is a modified from  'sig-alternate.cls'
%%%%%%%%%%%%%%%%%%%%%%%%%%%%%%%%%%%%

\documentclass{sig-alternate}
\setlength{\paperheight}{11in}
\setlength{\paperwidth}{8.5in}

\newcommand{\ignore}[1]{}
\usepackage[pass]{geometry}
\usepackage{fancyhdr}
%\usepackage[normalem]{ulem}
%\usepackage[hyphens]{url}
%\usepackage{hyperref}
\usepackage{color}
\usepackage{soul}

%%%%%%%%%%%---SETME-----%%%%%%%%%%%%%
\newcommand{\hpcasubmissionnumber}{6}
%%%%%%%%%%%%%%%%%%%%%%%%%%%%%%%%%%%%
%\sethlcolor{white}

%%%%%%%%%%% NADER: %%%%%%%%%%%%
\usepackage{xspace}
\usepackage{xcolor}
\usepackage{textcomp}
\usepackage{siunitx}
\usepackage[hyphens]{url}

\usepackage{listings}
\usepackage[breaklinks,colorlinks]{hyperref}
\hypersetup{citecolor=blue,linkcolor=blue}
%\def\sectionautorefname{Section}
%\def\subsectionautorefname{SubSection}

\renewcommand{\sectionautorefname}{\S}
\renewcommand{\subsectionautorefname}{\S}
%\usepackage[subpreambles=true]{standalone} 
\usepackage{pgfplots}
\usepackage{filecontents}
\usepackage{tikz}
\usetikzlibrary{arrows,calc}
\usetikzlibrary{patterns}
\usepackage{relsize}
\usepackage{tikz-cd}
%%
\usepackage{color}
\usepackage{enumitem}
\setlist{leftmargin=1em}

\definecolor{codegreen}{rgb}{0,0.6,0}
\definecolor{codegray}{rgb}{0.5,0.5,0.5}
\definecolor{codepurple}{rgb}{0.58,0,0.82}
\definecolor{backcolour}{rgb}{0.95,0.95,0.92}
\definecolor{backwhite}{rgb}{0.95,0.95,0.95}

\definecolor{mycolor}{rgb}{1,0,0}


\lstdefinestyle{mystyle}{
  backgroundcolor=\color{backwhite},   
  commentstyle=\color{codegreen},
  keywordstyle=\color{magenta},
  numberstyle=\tiny\color{backcolour},
  stringstyle=\color{codepurple},
  basicstyle=\footnotesize,
  breakatwhitespace=false,         
  breaklines=true,                 
  captionpos=b,                    
  keepspaces=true,                 
  numbers=left,                    
  numbersep=5pt,                  
  showspaces=false,                
  showstringspaces=false,
  showtabs=false,                  
  tabsize=2
}

\lstset{style=mystyle}

\renewcommand{\lstlistingname}{Example}

%%
\newcommand{\iot}{CPS\xspace} % IoT changed to CPS




% When sethlcolor is white, your highlights will not show up.  Use
% \sethlcolor{white} to submit your paper pdf.  When compiling your second
% pdf with highlighted changes, simply remove \sethlcolor{white} and add your
% optional 100-word appendix.
% Use \hl{ ... } to highlight any text.
%%%%%%%%%%%%%%%%%%%%%%%%%%%%%%%%%%%%

\fancypagestyle{firstpage}{
  \fancyhf{}
\setlength{\headheight}{50pt}
\renewcommand{\headrulewidth}{0pt}
  \fancyhead[C]{\normalsize{HPCA 2019 Industrial Track Submission
      \textbf{\#\hpcasubmissionnumber} -- Confidential Draft -- Do NOT Distribute!!}}
  \pagenumbering{arabic}
}

%%%%%%%%%%%---SETME-----%%%%%%%%%%%%%
\title{Understanding the Impact of ISAs on Performance and Power of Modern Embedded Systems}
%%%%%%%%%%%%%%%%%%%%%%%%%%%%%%%%%%%%

\begin{document}
\maketitle
\thispagestyle{firstpage}
\pagestyle{plain}



%%%%%% -- PAPER CONTENT STARTS-- %%%%%%%%

\begin{abstract}

  %% abstract


Instruction Set Architecture (ISA) is fundamental to how a wide variety of modern computer systems -- ranging from simple handheld mobile devices to large scale data centers -- are conceived, designed and implemented. One of the primary goals of ISA designers is to capture the most basic functions and tasks that can be used as basic building blocks to compose and express complex applications and softwares. The general expectation is that a computing system should perform these functions in most efficient manner in terms of performance, power, energy and area. 

While an ISA is central to computer design, there have been only a handful of successful ISAs till date. This limits designers' options and forces them to rely on a small subset even though it might not be efficient in capturing appropriate necessary functions needed for higher level applications. Unlike compilers, OSs, drivers, and other software components, ISAs have been a proprietary component by-and-large.

``Democratization of ISA'' was the main theme behind the advent of RISC-V with overarching goal to relieve the designer community and small- to mid-scale OEMs from the clutches of proprietary ISA suppliers. While this is a novel thought in spirit, in reality, much depends upon the efficacies of democratized ISA effort and the eventual ecosystem. In this work, we set out to discern and quantify the viability of an open source ISA such as RISC-V. We conduct numerous program and microarchitectural analysis on major RISC ISAs (ARM, MIPS and RISC-V) and present our findings. Overall, while RISC-V being a promising start so far, there is still a lot needs to be done to fully democratize ISA component.  

\vspace{0.5em}
\noindent
\textbf{Keywords:} RISC-V, ARM, MIPS, Performance, Power, Area, Energy, Program analysis, Microarchitecture.







\end{abstract}

\section{Introduction}

Numerous isas have been designed but couldn’t survive because they either didn’t offer anything unique or they were poorly designed.

RISCV is an emerging Instruction Set Architecture (ISA) and has become an important option for both academia and industry when considering new microprocessor designs. Features like modularity, extensibility, simplicity, and being open and free to use, make RISCV an attractive option for next generation of processors especially in embedded systems domain where new, customized, low-power, and efficient cores are needed.  In this paper we present a comparative study on impacts of three well-known Instruction Set Architectures (ISAs) (MIPS, ARM, and RISCV) on Performance, Power, and Area (PPA) of state-of-the-art embedded processors through a systematic measurement campaign using several different toolchains and frameworks and several standard benchmark suites.  We particularly study the impact of these ISAs on important metrics such as static and dynamic Instruction Count (\textit{icount}), Cycle Count, Microarchitectural Statistics (e.g. MPKI, Branch Prediction Accuracy, etc.), Dynamic Power, and Core's Area and report our key findings on impacts of using different ISAs on each of these metrics. We find that some of these metrics are ISA-dependent and others are dependent on other factors such as compiler, runtime libraries, and specific microarchitectural features. Our main conclusion is that while comparing to MIPS and ARM, RISCV has some shortcomings and design/toolchain issues that should be addressed and fixed, due to its intrinsic features such as modularity it provides a great opportunity for designing customized PPA-efficient cores.   

Instruction Set Architectures (ISAs) has a key role in designing cores for different domains, where x86 ISA has become dominant in desktop and server domains, and ARM has become the dominant ISA in mobile, tablet, and embedded system domain. The question of impact of ISA design on different Performance, Power, Area (PPA) metrics has traditionally been an important concern for designers and semiconductor industry especially in the 1980s and 1990s when
chip area and processor design complexity were the primary
constraints [24, 12, 17, 7]. In the past decade, we radical changes in computing landscape and rise of mobiles and tables and increasing popularity of ARM ISA this question again becomes an important issue. 

Today, with proliferation of embedded and cyber-physical systems (e.g. IoTs) and increasing popularity of domain-specific languages and emerging applications like machine-learning and more importantly, introduction of a new, open-source, modular ISA (RISCV), this question once again becomes an interesting topic for research. 

To answer this question, in this paper we present a comparative study on impacts of using three different ISAs (MIPS, ARM, and RISCV) on important metrics such as static and dynamic instruction count (icount), total execution time (cycle count), dynamic power, and area. We show which of these metrics are ISA-dependent and what are the other important factors on PPA. Using these experiments we pinpoint the shortcomings, issues, and advantages of using RISCV ISA over ARM and MIPS ISAs. 




\section{Methodology}
%
Numerous isas have been designed but couldn’t survive because they either didn’t offer anything unique or they were poorly designed.

RISCV is an emerging Instruction Set Architecture (ISA) and has become an important option for both academia and industry when considering new microprocessor designs. Features like modularity, extensibility, simplicity, and being open and free to use, make RISCV an attractive option for next generation of processors especially in embedded systems domain where new, customized, low-power, and efficient cores are needed.  In this paper we present a comparative study on impacts of three well-known Instruction Set Architectures (ISAs) (MIPS, ARM, and RISCV) on Performance, Power, and Area (PPA) of state-of-the-art embedded processors through a systematic measurement campaign using several different toolchains and frameworks and several standard benchmark suites.  We particularly study the impact of these ISAs on important metrics such as static and dynamic Instruction Count (\textit{icount}), Cycle Count, Microarchitectural Statistics (e.g. MPKI, Branch Prediction Accuracy, etc.), Dynamic Power, and Core's Area and report our key findings on impacts of using different ISAs on each of these metrics. We find that some of these metrics are ISA-dependent and others are dependent on other factors such as compiler, runtime libraries, and specific microarchitectural features. Our main conclusion is that while comparing to MIPS and ARM, RISCV has some shortcomings and design/toolchain issues that should be addressed and fixed, due to its intrinsic features such as modularity it provides a great opportunity for designing customized PPA-efficient cores.   

Instruction Set Architectures (ISAs) has a key role in designing cores for different domains, where x86 ISA has become dominant in desktop and server domains, and ARM has become the dominant ISA in mobile, tablet, and embedded system domain. The question of impact of ISA design on different Performance, Power, Area (PPA) metrics has traditionally been an important concern for designers and semiconductor industry especially in the 1980s and 1990s when
chip area and processor design complexity were the primary
constraints [24, 12, 17, 7]. In the past decade, we radical changes in computing landscape and rise of mobiles and tables and increasing popularity of ARM ISA this question again becomes an important issue. 

Today, with proliferation of embedded and cyber-physical systems (e.g. IoTs) and increasing popularity of domain-specific languages and emerging applications like machine-learning and more importantly, introduction of a new, open-source, modular ISA (RISCV), this question once again becomes an interesting topic for research. 

To answer this question, in this paper we present a comparative study on impacts of using three different ISAs (MIPS, ARM, and RISCV) on important metrics such as static and dynamic instruction count (icount), total execution time (cycle count), dynamic power, and area. We show which of these metrics are ISA-dependent and what are the other important factors on PPA. Using these experiments we pinpoint the shortcomings, issues, and advantages of using RISCV ISA over ARM and MIPS ISAs. 




\section{Background}

\label{sec:bak}
RISC-V is an emerging open-source software and hardware ecosystem that has gained in popularity in both industry and academia [2, 11]. At the heart of the ecosystem, the RISC-V ISA is designed to be open, simple, extensible, and free to use. The RISC-V software tool chain includes open-source compilers (e.g., GNU/GCC and LLVM), a full Linux port, a GNU/GDB debugger, verification tools, and simulators. On the hardware side, several RISC-V prototypes (e.g., Celerity [4]) have been published. The rapid growth of the RISC-V ecosystem enables computer architects to quickly leverage RISC-V in their research.




\section{Related Work}
%Early ISA studies are instructive but miss key changes in today’s
microprocessors and design constraints that have shifted the ISA’s effect. We review
previous comparisons in chronological order and observe that all prior comprehensive
ISA studies considering commercially implemented processors focused exclusively on
performance.
Bhandarkar and Clark compared theMIPS and VAX ISA by comparing theM/2000 to
the Digital VAX 8700 implementations [Bhandarkar and Clark 1991] and concluded:
“RISC as exemplified by MIPS provides a significant processor performance advantage.”
In another study in 1995, Bhandarkar compared the Pentium-Pro to the Alpha
21164 [Bhandarkar 1997], again focused exclusively on performance and concluded:
“the Pentium Pro processor achieves 80\% to 90\% of the performance of the Alpha
21164... It uses an aggressive out-of-order design to overcome the instruction set level
limitations of a CISC architecture. On floating-point intensive benchmarks, the Alpha
21164 does achieve over twice the performance of the Pentium Pro processor.” Consensus
had grown that RISC and CISC ISAs had fundamental differences that led to
performance gaps that required aggressive microarchitecture optimization for CISC
that only partially bridged the gap.
Isen et al. [2009] compared the performance of Power5+ to Intel Woodcrest considering
SPEC benchmarks and concluded that x86 matches the POWER ISA. The
consensus was that “with aggressive microarchitectural techniques for ILP, CISC and
RISC ISAs can be implemented to yield very similar performance.”
Many informal studies in recent years claim the x86’s “crufty” CISC ISA incursmany
power overheads and attribute the ARM processor’s power efficiency to the ISA.1 These
studies suggest that the microarchitecture optimizations from the past decades have
led to RISC and CISC cores with similar performance but that the power overheads of
CISC are intractable.
In light of the ISA studies from decades past, the significantly modified computing
landscape, and the seemingly vastly different power consumption of RISC implementations
(ARM: 1–2W, MIPS: 1–4W) to CISC implementations (x86: 5–36W), we feel there
is need to revisit this debate with a rigorous methodology. Specifically, considering the
multipronged importance of the metrics of power, energy, and performance, we need to
compare RISC to CISC on those three metrics. Macro-op cracking and decades of research
in high-performance microarchitecture techniques and compiler optimizations
seemingly help overcome x86’s performance and code-effectiveness bottlenecks, but these approaches are not free. The crux of our analysis is the following: After decades
of research to mitigate CISC performance overheads, do the new approaches introduce
fundamental energy inefficiencies?


% This document provides instructions for submitting papers to the 25th
% International Symposium on High Performance Computer Architecture (HPCA), 2019.  In an
% effort to respect the efforts of reviewers and in the interest of
% fairness to all prospective authors, we request that all submissions
% to HPCA 2019 follow the formatting and submission rules detailed
% below. Submissions that violate these instructions may not be reviewed,
% at the discretion of the program chairs, in order to maintain a review
% process that is fair to all potential authors.


% An example file (formatted using the HPCA'19 submission format) that
% contains the formatting guidelines can be downloaded from the 
% \href{http://hpca2019.seas.gwu.edu/}{HPCA webpage}.  The content of this
% document mirrors the submission instructions that appear on that
% page.  The link to the HotCRP submission site also appears on the HPCA
% webpage.

% All questions regarding paper formatting and submission should be directed
% to the program co-chairs using the email address hpca19@cs.utah.edu.

% \subsection{Format Highlights}
% Note that there are several notices for paper format:
% \begin{itemize}
% \item Paper must be submitted in printable PDF format.
% \item Text must be in a minimum 10pt ({\bf not} 9pt) font.
% \item Papers must be at most 11 pages, not including references. 
% \item No page limit for references.
% \item The references must include complete author lists (no {\em et al.}).
% \end{itemize}


% \subsection{Paper Evaluation Objectives}
% The committee will make every effort to judge each submitted paper on
% its own merits. There will be no target acceptance rate.
% We expect to accept a wide range of papers with appropriate expectations
% for evaluation --- while papers that build on significant past work
% with strong evaluations are valuable, papers that open new areas with
% less rigorous evaluation are equally welcome and especially encouraged.
% Given the wide range of topics covered by HPCA, every effort will be
% made to find expert reviewers.

% \subsection{Optional Second PDF for Resubmitted Papers}

% For submissions that were previously submitted to other conferences,
% we strongly encourage authors to take into account feedback received
% from previous reviews.  There is a reasonable probability that an
% HPCA-25 submission may be assigned to reviewers that have seen earlier
% versions of the same paper.  
% \hl{To ease the reviewing burden, and to help authors highlight the
% improvements to their paper, authors can optionally submit a second
% pdf that highlights the major changes to their work, relative to prior
% submissions.}
% We anticipate that this process will hold authors and reviewers more
% accountable, and perhaps help reduce any pre-conceived bias that a
% reviewer may have against the paper.
% \begin{itemize}
% \item The second pdf will not be made visible to reviewers by default.  If a reviewer believes they have reviewed an earlier version of the paper, they will email the PC Chairs and ask for the second pdf.
% \item The second pdf can not introduce new content.  It must be identical to the HPCA-25 submission, but can use color to highlight specific text/captions that are different from prior versions of the paper.  A 100-word appendix may be included to summarize and provide context for the major changes.
% \item In latex, text can be highlighted using \hl{ \textbackslash hl\{...\}, as has been done here.}  The submission template already includes the necessary latex packages color and soul.  To remove the highlight, simply set the highlight color to white (example in the latex template).
% \item Authors have the flexibility to decide what significant changes they want to highlight to a reviewer.  Obviously, authors will want to draw attention to all prior show-stopping concerns that have been addressed in the new submission.
% \item The second pdf upload deadline is also at 11:59pm EDT on Friday August 3rd.  For those planning on working on their paper until the last minute, a one hour grace period is being offered.
% \item In case of any questions about this new policy, please feel free to email the PC chairs at: \\ hpca19@cs.utah.edu.
% \end{itemize}


% \section{Paper Preparation Instructions}

% \subsection{Paper Formatting}

% Papers must be submitted in printable PDF format and should contain a
% {\bf maximum of 11 pages} of single-spaced two-column text, {\bf not
%   including references}.  You may include any number of pages for
% references, but see below for more instructions.  If you are using
% \LaTeX~\cite{lamport94} to typeset your paper, then we suggest that
% you use the template here:
% \href{http://hpca2019.seas.gwu.edu//hpca25-latex-template.tar.gz}{\LaTeX~Template}. This
% document was prepared with that template.  If you use a different
% software package to typeset your paper, then please adhere to the
% guidelines given in Table~\ref{table:formatting}.

% \begin{scriptsize}
% \begin{table}[h!]
%   \centering
%   \begin{tabular}{|l|l|}
%     \hline
%     \textbf{Field} & \textbf{Value}\\
%     \hline
%     \hline
%     File format & PDF \\
%     \hline
%     Page limit & 11 pages, {\bf not including}\\
%                & {\bf references}\\
%     \hline
%     Paper size & US Letter 8.5in $\times$ 11in\\
%     \hline
%     Top margin & 1in\\
%     \hline
%     Bottom margin & 1in\\
%     \hline
%     Left margin & 0.75in\\
%     \hline
%     Right margin & 0.75in\\
%     \hline
%     Body & 2-column, single-spaced\\
%     \hline
%     Space between columns & 0.25in\\
%     \hline
%     Body font & 10pt\\
%     \hline
%     Abstract font & 10pt, italicized\\
%     \hline
%     Section heading font & 12pt, bold\\
%     \hline
%     Subsection heading font & 10pt, bold\\
%     \hline
%     Caption font & 9pt (minimum), bold\\
%     \hline
%     References & 8pt, no page limit, list \\
%                & all authors' names\\
%     \hline
%   \end{tabular}
%   \caption{Formatting guidelines for submission. }
%   \label{table:formatting}
% \end{table}
% \end{scriptsize}

% \textbf{Please ensure that you include page numbers with your
% submission}. This makes it easier for the reviewers to refer to different
% parts of your paper when they provide comments.

% Please ensure that your submission has a banner at the top of the
% title page, similar to
% \href{http://hpca2019.seas.gwu.edu/hpca25_sample.pdf}{this one},
% which contains the submission number and the notice of
% confidentiality.  If using the template, just replace XXX with your
% submission number.

% \subsection{Content}

% %\noindent\textbf{\sout{Author List.}} Reviewing will be double blind;
% therefore, please do not include any author names on any submitted
% documents except in the space provided on the submission form.  You must
% also ensure that the metadata included in the PDF does not give away the
% authors. If you are improving upon your prior work, refer to your prior
% work in the third person and include a full citation for the work in the
% bibliography.  For example, if you are building on {\em your own} prior
% work in the papers \cite{nicepaper1,nicepaper2,nicepaper3}, you would say
% something like: ``While the authors of
% \cite{nicepaper1,nicepaper2,nicepaper3} did X, Y, and Z, this paper
% additionally does W, and is therefore much better."  Do NOT omit or
% anonymize references for blind review.  There is one exception to this: for
% your own prior work that appeared in IEEE CAL or workshops without archived
% proceedings, as discussed later in this document.

% \noindent\textbf{Figures and Tables.} Ensure that the figures and tables
% are legible.  Please also ensure that you refer to your figures in the main
% text.  Many reviewers print the papers in gray-scale. Therefore, if you use
% colors for your figures, ensure that the different colors are highly
% distinguishable in gray-scale.

% \noindent\textbf{References.}  There is no length limit for references.
% {\bf Each reference must explicitly list all authors of the paper.  Papers
% not meeting this requirement will be rejected.} Authors of NSF proposals
% should be familiar with this requirement. Knowing all authors of related
% work will help find the best reviewers. Since there is no length limit
% for the number of pages used for references, there is no need to save space
% here.

% \section{Paper Submission Instructions}

% \subsection{Guidelines for Determining Authorship}


% IEEE guidelines dictate that authorship should be based on a {\bf
%   substantial intellectual contribution}. It is assumed that all
% authors have had a significant role in the creation of an article that
% bears their names. In particular, the authorship credit must be
% reserved only for individuals who have met each of the following
% conditions:

% \begin{enumerate}

% \item Made a significant intellectual contribution to the theoretical
%   development, system or experimental design, prototype development,
%   and/or the analysis and interpretation of data associated with the
%   work contained in the article;

% \item Contributed to drafting the article or reviewing and/or revising
%   it for intellectual content; and

% \item Approved the final version of the article as accepted for
%   publication, including references.

% \end{enumerate}

% A detailed description of the IEEE authorship guidelines and
% responsibilities is available
% \href{https://www.ieee.org/publications_standards/publications/rights/Section821.html}{here}.
% Per these guidelines, it is not acceptable to award {\em honorary }
% authorship or {\em gift} authorship. Please keep these guidelines in
% mind while determining the author list of your paper.


% \subsection{Declaring Authors}

% Declare all the authors of the paper upfront. Addition/removal of authors
% once the paper is accepted will have to be approved by the program co-chairs,
% since it potentially undermines the goal of eliminating conflicts for
% reviewer assignment.


% \subsection{Areas and Topics}

% Authors should indicate these areas on the submission form as
% well as specific topics covered by the paper for optimal reviewer match. If
% you are unsure whether your paper falls within the scope of HPCA, please
% check with the program chair -- HPCA is a broad, multidisciplinary
% conference and encourages new topics.

% \subsection{Declaring Conflicts of Interest}

% Authors must register all their conflicts on the paper submission site.
% Conflicts are needed to ensure appropriate assignment of reviewers.
% If a paper is found to have an undeclared conflict that causes
% a problem OR if a paper is found to declare false conflicts in order to
% abuse or ``game'' the review system, the paper may be rejected.

% Pease declare a conflict of interest (COI) with the following people for any author of your paper:

% \begin{enumerate}
% \item Your Ph.D. advisor(s), post-doctoral advisor(s), Ph.D. students,
%       and post-doctoral advisees, forever.
% \item Family relations by blood or marriage, or their equivalent,
%       forever (if they might be potential reviewers).
% \item People with whom you have collaborated in the last FIVE years, including
% \begin{itemize}
% \item co-authors of accepted/rejected/pending papers.
% \item co-PIs on accepted/rejected/pending grant proposals.
% \item funders (decision-makers) of your research grants, and researchers
%       whom you fund.
% \end{itemize}
% \item People (including students) who shared your primary institution(s) in the
% last FIVE years.
% \item Other relationships, such as close personal friendship, that you think might tend
% to affect your judgment or be seen as doing so by a reasonable person familiar
% with the relationship.
% \end{enumerate}

% ``Service'' collaborations such as co-authoring a report for a professional
% organization, serving on a program committee, or co-presenting
% tutorials, do not themselves create a conflict of interest.
% Co-authoring a paper that is a compendium of various projects with
% no true collaboration among the projects does not constitute a
% conflict among the authors of the different projects.

% On the other hand, there may be others not covered by the above with
% whom you believe a COI exists, for example, an ongoing collaboration
% which has not yet resulted in the creation of a paper or proposal.
% Please report such COIs; however, you may be asked to justify them.
% Please be reasonable. For example, you cannot declare a COI with a
% reviewer just because that reviewer works on topics similar to or
% related to those in your paper.  The PC Chair may contact co-authors
% to explain a COI whose origin is unclear.

% We hope to draw most reviewers from the PC and the ERC, but others from the
% community may also write reviews.  Please declare all your conflicts (not
% just restricted to the PC and ERC).  When in doubt, contact the program
% co-chairs.

% \subsection{Concurrent Submissions and Workshops}

% By submitting a manuscript to HPCA'19, the authors guarantee that the
% manuscript has not been previously published or accepted for publication in
% a substantially similar form in any conference, journal, or the archived
% proceedings of a workshop (e.g., in the ACM digital library) -- see
% exceptions below. The authors also guarantee that no paper that contains
% significant overlap with the contributions of the submitted paper will be
% under review for any other conference or journal or an archived proceedings
% of a workshop during the HPCA'19 review period. Violation of any of these
% conditions will lead to rejection.

% The only exceptions to the above rules are for the authors' own papers
% in (1) workshops without archived proceedings such as in the ACM digital
% library (or where the authors chose not to have their paper appear in the
% archived proceedings), or (2) venues such as IEEE CAL where there is an
% explicit policy that such publication does not preclude longer conference
% submissions.  In all such cases, the submitted manuscript may ignore the above
% work to preserve author anonymity. This information must, however, be provided
% on the submission form -- the PC chairs will make this information available to
% reviewers if it becomes necessary to ensure a fair review.  As always, if you
% are in doubt, it is best to contact the program co-chairs.
        
% Authors are not barred from uploading their papers to arxiv and similar sites.
% But please note that such efforts may compromise the double-blind review
% process, so please exercise care when discussing your submission in public
% forums.  On a related note, unrefereed on-line pre-prints are not assumed to
% constitute "prior work" -- in other words, reviewers cannot penalize an HPCA
% submission because it does not cite a pre-print with limited visibility.


% Finally, we also note that the IEEE Plagiarism Guidelines \href{http://www.ieee.org/publications\_standards/publications/rights/plagiarism.html}{IEEE Plagiarism Guidelines} covers a range of ethical issues concerning the misrepresentation of other works or
% one's own work.

% \section{Acknowledgements}
% This document is derived from previous conferences, in particular HPCA 2017 and HPCA 2018. 


%%%%%%% -- PAPER CONTENT ENDS -- %%%%%%%%


%%%%%%%%% -- BIB STYLE AND FILE -- %%%%%%%%
\bibliographystyle{ieeetr}
\bibliography{ref}
%%%%%%%%%%%%%%%%%%%%%%%%%%%%%%%%%%%%

\end{document}
