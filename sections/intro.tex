
Numerous isas have been designed but couldn’t survive because they either didn’t offer anything unique or they were poorly designed.

RISCV is an emerging Instruction Set Architecture (ISA) and has become an important option for both academia and industry when considering new microprocessor designs. Features like modularity, extensibility, simplicity, and being open and free to use, make RISCV an attractive option for next generation of processors especially in embedded systems domain where new, customized, low-power, and efficient cores are needed.  In this paper we present a comparative study on impacts of three well-known Instruction Set Architectures (ISAs) (MIPS, ARM, and RISCV) on Performance, Power, and Area (PPA) of state-of-the-art embedded processors through a systematic measurement campaign using several different toolchains and frameworks and several standard benchmark suites.  We particularly study the impact of these ISAs on important metrics such as static and dynamic Instruction Count (\textit{icount}), Cycle Count, Microarchitectural Statistics (e.g. MPKI, Branch Prediction Accuracy, etc.), Dynamic Power, and Core's Area and report our key findings on impacts of using different ISAs on each of these metrics. We find that some of these metrics are ISA-dependent and others are dependent on other factors such as compiler, runtime libraries, and specific microarchitectural features. Our main conclusion is that while comparing to MIPS and ARM, RISCV has some shortcomings and design/toolchain issues that should be addressed and fixed, due to its intrinsic features such as modularity it provides a great opportunity for designing customized PPA-efficient cores.   

Instruction Set Architectures (ISAs) has a key role in designing cores for different domains, where x86 ISA has become dominant in desktop and server domains, and ARM has become the dominant ISA in mobile, tablet, and embedded system domain. The question of impact of ISA design on different Performance, Power, Area (PPA) metrics has traditionally been an important concern for designers and semiconductor industry especially in the 1980s and 1990s when
chip area and processor design complexity were the primary
constraints [24, 12, 17, 7]. In the past decade, we radical changes in computing landscape and rise of mobiles and tables and increasing popularity of ARM ISA this question again becomes an important issue. 

Today, with proliferation of embedded and cyber-physical systems (e.g. IoTs) and increasing popularity of domain-specific languages and emerging applications like machine-learning and more importantly, introduction of a new, open-source, modular ISA (RISCV), this question once again becomes an interesting topic for research. 

To answer this question, in this paper we present a comparative study on impacts of using three different ISAs (MIPS, ARM, and RISCV) on important metrics such as static and dynamic instruction count (icount), total execution time (cycle count), dynamic power, and area. We show which of these metrics are ISA-dependent and what are the other important factors on PPA. Using these experiments we pinpoint the shortcomings, issues, and advantages of using RISCV ISA over ARM and MIPS ISAs. 


