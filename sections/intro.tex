Instruction Set Architectures (ISAs) has a key role in designing cores for different domains, where x86 ISA has become dominant in desktop and server domains, and ARM has become the dominant ISA in mobile, tablet, and embedded system domain. The question of impact of ISA design on different Performance, Power, Area (PPA) metrics has traditionally been an important concern for designers and semiconductor industry especially in the 1980s and 1990s when
chip area and processor design complexity were the primary
constraints [24, 12, 17, 7]. In the past decade, we radical changes in computing landscape and rise of mobiles and tables and increasing popularity of ARM ISA this question again becomes an important issue. 

Today, with proliferation of embedded and cyber-physical systems (e.g. IoTs) and increasing popularity of domain-specific languages and emerging applications like machine-learning and more importantly, introduction of a new, open-source, modular ISA (RISCV), this question once again becomes an interesting topic for research. 

To answer this question, in this paper we present a comparative study on impacts of using three different ISAs (MIPS, ARM, and RISCV) on important metrics such as static and dynamic instruction count (icount), total execution time (cycle count), dynamic power, and area. We show which of these metrics are ISA-dependent and what are the other important factors on PPA. Using these experiments we pinpoint the shortcomings, issues, and advantages of using RISCV ISA over ARM and MIPS ISAs. 


