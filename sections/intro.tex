\label{sec:intro}

In past few decades, numerous ISAs have been proposed and designed but only a handful of them proliferated to make a real impact. In the long series of ISA invention, RISC-V is an emerging ISA and is increasingly becoming an important option for both academia and industry when considering new microprocessor designs~\cite{risc-v}. Features like modularity, extensibility, simplicity, and being open and free to use, make RISC-V an attractive option for next generation of processors especially in embedded systems domain where new, customized, low-power, and efficient cores are needed.  

ISA component plays a key role in designing and customizing cores for different domains. x86 ISA has become dominant in desktop and server domains~\cite{x86-isa}, and ARM has become the dominant ISA in mobile, tablet, and embedded system domain~\cite{arm-isa}. \emph{``What impact an ISA has on different metrics such as performance, power, area (PPA)?''} has traditionally been an important concern for designers and semiconductor industry in the 80s and 90s when chip area and processor design complexity were the primary constraints [24, 12, 17, 7]. In the recent past, last decade or so, radical changes in computing landscape and rise of mobiles, tablets and increasing popularity of ARM ISA have raised this question again and has become an important issue. 

Today, with proliferation of embedded and cyber-physical systems (e.g., IoTs) and increasing popularity of domain-specific languages and emerging applications like machine-learning and more importantly, introduction of a new, open-source, modular ISA (RISC-V), this question once again becomes an interesting topic for research. In this work, we set out to show which of the well known metrics (i.e., PPA) are ISA-dependent or ISA-influenced. We also try to address other factors such as differences in toolchains and mature ecosystem that can impact PPA metrics greatly. 

In this paper, we present a comparative study of three well-known ISAs (MIPS, ARM, and RISCV) and their impact on PPA in the state-of-the-art embedded processors through a systematic measurement campaign using several different toolchains and frameworks. Using these experiments, we pinpoint the shortcomings, issues, and advantages of using RISC-V ISA over ARM and MIPS ISAs.  In particular, we study the impact of these ISAs on important metrics such as static and dynamic instruction count (\textit{icount}), total cycles to execute a program (\textit{cycle\_count}), microarchitectural statistics (cache and branch MPKI, branch prediction accuracy, etc.), dynamic power, and core's area. We find that some of these are ISA-dependent and other metrics are dependent on other factors such as compiler, runtime libraries, and specific microarchitectural features. 

Our primary observation is that while comparing to MIPS and ARM, RISC-V has some shortcomings and design/toolchain issues that should be addressed and fixed to make it more competitive. On the positive side, due to its intrinsic features such as modularity RISC-V provides a great opportunity for designing customized PPA-efficient cores. 

Rest of the paper is organized as follows: Section~\ref{sec:bak} presents an overview of the background to our research. Section~\ref{sec:method} details out methodology. Section~\ref{sec:expt} presents our experimental setup and detailed results. Finally, section~\ref{sec:related} presents an overview of related work before we conclude in Section~\ref{sec:conc}.



