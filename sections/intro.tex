\label{sec:intro}

In past few decades, numerous ISAs have been proposed and designed but only a handful of them proliferated to make a real impact. In the long series of ISA invention, RISC-V is an emerging ISA and is increasingly becoming an important option for both academia and industry when considering new microprocessor designs~\cite{risc-v}. Features like modularity, extensibility, simplicity, and being open and free to use, make RISC-V an attractive option for next generation of processors especially in embedded systems domain where new, customized, low-power, and efficient cores are needed.  

ISA has a key role in designing cores for different domains. x86 ISA has become dominant in desktop and server domains~\cite{x86-isa}, and ARM has become the dominant ISA in mobile, tablet, and embedded system domain~\cite{arm-isa}. The question of impact of ISA design on different Performance, Power, Area (PPA) metrics has traditionally been an important concern for designers and semiconductor industry especially in the 1980s and 1990s when chip area and processor design complexity were the primary
constraints [24, 12, 17, 7]. In the past decade, radical changes in computing landscape and rise of mobiles, tables and increasing popularity of ARM ISA this question again becomes an important issue. 

Today, with proliferation of embedded and cyber-physical systems (e.g. IoTs) and increasing popularity of domain-specific languages and emerging applications like machine-learning and more importantly, introduction of a new, open-source, modular ISA (RISCV), this question once again becomes an interesting topic for research. We show which of the well known metrics are ISA-dependent and what are the other important factors that impact PPA. Using these experiments we pinpoint the shortcomings, issues, and advantages of using RISC-V ISA over ARM and MIPS ISAs.  
 

In this paper, we present a comparative study on impacts of three well-known ISAs (MIPS, ARM, and RISCV) on performance, power, and area (PPA) on state-of-the-art embedded processors through a systematic measurement campaign using several different toolchains and frameworks. In particular, we study the impact of these ISAs on important metrics such as static and dynamic instruction count (\textit{icount}), total cycles to execute a program, microarchitectural statistics (e.g. MPKI, branch prediction accuracy, etc.), dynamic power, and core's area. We report our key findings on impacts of using different ISAs on each of these metrics. We find that some of these are ISA-dependent and other metrics are dependent on other factors such as compiler, runtime libraries, and specific microarchitectural features. 

Our primary observation is that while comparing to MIPS and ARM, RISC-V has some shortcomings and design/toolchain issues that should be addressed and fixed to make it more competitive. On the positive side, due to its intrinsic features such as modularity RISC-V provides a great opportunity for designing customized PPA-efficient cores. 

Rest of the paper is organized as follows: Section~\ref{sec:bak} presents an overview of the background to our research. Section~\ref{sec:method} details out methodology. Section~\ref{sec:expt} presents our experimental setup and detailed results. Finally, section~\ref{sec:related} presents an overview of related work before we conclude in Section~\ref{sec:conc}.



