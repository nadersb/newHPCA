\label{sec:related}

Early ISA studies are instructive but miss key changes in today’s
microprocessors and design constraints that have shifted the ISA’s effect. We review
previous comparisons in chronological order and observe that all prior comprehensive
ISA studies considering commercially implemented processors focused exclusively on
performance.

Bhandarkar and Clark compared the MIPS and VAX ISA by comparing the M/2000 to
the Digital VAX 8700 implementations [Bhandarkar and Clark 1991] and concluded:
“RISC as exemplified by MIPS provides a significant processor performance advantage.”
In another study in 1995, Bhandarkar compared the Pentium-Pro to the Alpha
21164 [Bhandarkar 1997], again focused exclusively on performance and concluded:
“the Pentium Pro processor achieves 80\% to 90\% of the performance of the Alpha
21164. It uses an aggressive out-of-order design to overcome the instruction set level
limitations of a CISC architecture. On floating-point intensive benchmarks, the Alpha
21164 does achieve over twice the performance of the Pentium Pro processor.” 

Consensus had grown that RISC and CISC ISAs had fundamental differences that led to
performance gaps that required aggressive microarchitecture optimization for CISC
that only partially bridged the gap. Isen et al. [2009] compared the performance of Power5+ to Intel Woodcrest considering SPEC benchmarks and concluded that x86 matches the POWER ISA. 

The consensus was that “with aggressive microarchitectural techniques for ILP, CISC and
RISC ISAs can be implemented to yield very similar performance.” Many informal studies in recent years claim the x86’s “crufty” CISC ISA incurs many power overheads and attribute the ARM processor’s power efficiency to the ISA. These studies suggest that the microarchitecture optimizations from the past decades have led to RISC and CISC cores with similar performance but that the power overheads of CISC are intractable.

In light of the ISA studies from decades past, the significantly modified computing
landscape, and the seemingly vastly different power consumption of RISC implementations
(ARM: 1–2W, MIPS: 1–4W) to CISC implementations (x86: 5–36W), we feel there
is need to revisit this debate with a rigorous methodology. Specifically, considering the
multipronged importance of the metrics of power, energy, and performance, we need to
compare RISC to CISC on those three metrics. Macro-op cracking and decades of research
in high-performance microarchitecture techniques and compiler optimizations
seemingly help overcome x86’s performance and code-effectiveness bottlenecks, but these approaches are not free and might have significant overheads. 