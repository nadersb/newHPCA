
\label{sec:bak}
RISC-V is an emerging open-source software and hardware ecosystem that has gained in popularity in both industry and academia [2, 11]. At the heart of the ecosystem, the RISC-V ISA is designed to be open, simple, extensible, and free to use. The RISC-V software tool chain includes open-source compilers (e.g., GNU/GCC and LLVM), a full Linux port, a GNU/GDB debugger, verification tools, and simulators. On the hardware side, several RISC-V prototypes (e.g., Celerity [4]) have been published. The rapid growth of the RISC-V ecosystem enables computer architects to quickly leverage RISC-V in their research.

RISC-V ISA has a modular design and defines parts of its ISA as extensions that are coded with a letter and written as ``RV32I'' or ``RV64IMAFD''. There are small differences in the same extension for different register sizes. If the architecture (32/64 bit) is not important it can be left out and described as "RVI" or "RVIMAFD". The ISA Base and its extensions are developed in a collective effort between industry, the research community and educational institutions. Extensions marked as "frozen" are not expected to change in any way except clarifications and improvements in documentation.

The base set of RISC-V is the Base Integer Instruction Set "RV32/64/128I" or "RV32E" (a reduced version of RV32I that supports only 16 registers designed for embedded systems). This set by itself can implement a simplified general-purpose computer, with full software support, including a general-purpose compiler.[4]

A computer design may add further extensions: Integer multiplication and division ("M"), Atomic instructions ("A") for handling real-time concurrency, IEEE Floating point ("F") with Double-precision ("D") and Quad-precision ("Q") options.[4] There's also an optional "compact" ("C") extension to reduce code size. Many RISC-V computers might add the compact extension to reduce power consumption, code size, and memory usage.[4]

There are future plans for to support hypervisors, virtualization,[18] bit-manipulation ("B"), decimal floating-point ("L"), Packed SIMD (i.e. budget multimedia, "P"), vector processing ("V") and transactional memory ("T").[4] Below is an overview with all extensions that are finalized (frozen) and currently in development.

A small 32-bit computer for an embedded system might be "RV32EC". A large 64-bit computer might be "RV64IMAFDC". A computer with the instruction sets "IMAFD", an "RV32/64IMAFD", is said to be "general-purpose", summarized as "G,"[4] so an "RV64IMAFDC" can also be described as an "RV64GC". Together with the "privileged" instruction set extension an "RVGC" defines all instructions needed to conveniently support a Unix-style operating system.


