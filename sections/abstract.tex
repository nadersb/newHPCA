%% abstract


Instruction Set Architecture (ISA) is fundamental to how a wide variety of modern computer systems -- ranging from simple handheld mobile devices to large scale data centers -- are conceived, designed and implemented. One of the primary goals of ISA designers is to capture the most basic functions and tasks that can be used as basic building blocks to compose and express complex applications and softwares. The general expectation is that a computing system should perform these functions in most efficient manner in terms of performance, power, energy and area. 

While an ISA is central to computer design, there have been only a handful of successful ISAs till date. This limits designers' options and forces them to rely on a small subset even though it might not be efficient in capturing appropriate necessary functions needed for higher level applications. Unlike compilers, OSs, drivers, and other software components, ISAs have been a proprietary component by-and-large.

``Democratization of ISA'' was the main theme behind the advent of RISC-V with overarching goal to relieve the designer community and small- to mid-scale OEMs from the clutches of proprietary ISA suppliers. While this is a novel thought in spirit, in reality, much depends upon the efficacies of democratized ISA effort and the eventual ecosystem. In this work, we set out to discern and quantify the viability of an open source ISA such as RISC-V. We conduct numerous program and microarchitectural analysis on major RISC ISAs (ARM, MIPS and RISC-V) and present our findings. Overall, while RISC-V being a promising start so far, there is still a lot needs to be done to fully democratize ISA component.  

\vspace{0.5em}
\noindent
\textbf{Keywords:} RISC-V, ARM, MIPS, Performance, Power, Area, Energy, Program analysis, Microarchitecture.





