%% abstract


Instruction Set Architecture (ISA) is fundamental to how a wide variety of modern computer systems -- ranging from simple handheld mobile devices to large scale data centers and server farms -- are conceived, designed and implemented. ISA designers' goal is often to capture the most basic functions and tasks that can be then used to compose and express complex applications and softwares. The General expectation is that a computing system should perform functions captured by ISA set in most possible efficient way in terms of performance (single-cycle) and power/energy. 

While an ISA is central to computer design, there have been only a handful of successful ISAs forcing designers to choose them from a small subset even though it might not be the most efficient in capturing higher level applications. Unlike compilers, OSs, drivers, and other software components, ISAs have been a proprietary component by-and-large.

Democratization of ISA was the main theme behind the advent of RISC-V. It was touted to relieve the designer community and small to mid-scale OEMs from the clutches of proprietary ISA suppliers. While this is a novel thought in spirit, much depends upon the efficacies of democratized ISA itself. In this work, we set out to discern and quantify the viability of an open source ISA such as RISC-V. 





