%% abstract


Instruction Set Architectures (ISA) are fundamental to how a wide variety of modern days computer systems -- ranging from simple hand held mobile devices to large scale data centers and server farms -- are conceived, designed and implemented. Often, ISA designer goal is to capture most basic functions and tasks that can be then used to compose complex applications and softwares. Expectation is that a computing system should perform functions captured by ISA set in most possible efficient way in terms of performance (single-cycle) and power/energy. While an ISA is central to computer design, there have been only a handful of successful ISAs forcing designers to choose them from a small subset even though it might not be the most efficient in capturing higher level applications. Unlike compilers, OSs, drivers, and other software components, ISAs have been a proprietary component by-and-large.

Democratization of ISA was the main theme behind the advent of RISC-V. It was touted to relieve the designer community and small to mid-scale OEMs from the clutches of proprietary ISA suppliers. While this is a novel thought in spirit, much depends upon the efficacies of democratized ISA itself. In this work, we set out to discern and quantify the viability of an open source ISA such as RISC-V. 

To best of our knowledge, this is first work to compare RISC-V with its popular proprietary counterparts (ARM and MIPS). We used state-of-art simulation and emulation frameworks: \textit{qemu} for program analysis and \textit{gem5} for microarchitectural simulations. We also present concrete cases where RISC-V clearly falls behind compared to ARM, MIPS ISAs and what addendum could possibly make it competitive. To our surprise, we also stumbled upon cases where RISC-V is better than proprietary ISAs. Overarching goal of our exploration is to enable RISC-V designers so that they can augment their designs and be able to close the gap with other state-of-art ISAs.



