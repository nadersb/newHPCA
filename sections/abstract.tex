%% abstract
RISCV is an emerging Instruction Set Architecture (ISA) and has become an important option for both academia and industry when considering new microprocessor designs. Features like modularity, extensibility, simplicity, and being open and free to use, make RISCV an attractive option for next generation of processors especially in embedded systems domain where new, customized, low-power, and efficient cores are needed.  In this paper we present a comparative study on impacts of three well-known Instruction Set Architectures (ISAs) (MIPS, ARM, and RISCV) on Performance, Power, and Area (PPA) of state-of-the-art embedded processors through a systematic measurement campaign using several different toolchains and frameworks and several standard benchmark suites.  We particularly study the impact of these ISAs on important metrics such as static and dynamic Instruction Count (\textit{icount}), Cycle Count, Microarchitectural Statistics (e.g. MPKI, Branch Prediction Accuracy, etc.), Dynamic Power, and Core's Area and report our key findings on impacts of using different ISAs on each of these metrics. We find that some of these metrics are ISA-dependent and others are dependent on other factors such as compiler, runtime libraries, and specific microarchitectural features. Our main conclusion is that while comparing to MIPS and ARM, RISCV has some shortcomings and design/toolchain issues that should be addressed and fixed, due to its intrinsic features such as modularity it provides a great opportunity for designing customized PPA-efficient cores.   


